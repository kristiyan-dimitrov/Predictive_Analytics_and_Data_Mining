\documentclass[]{article}
\usepackage{lmodern}
\usepackage{amssymb,amsmath}
\usepackage{ifxetex,ifluatex}
\usepackage{fixltx2e} % provides \textsubscript
\ifnum 0\ifxetex 1\fi\ifluatex 1\fi=0 % if pdftex
  \usepackage[T1]{fontenc}
  \usepackage[utf8]{inputenc}
\else % if luatex or xelatex
  \ifxetex
    \usepackage{mathspec}
  \else
    \usepackage{fontspec}
  \fi
  \defaultfontfeatures{Ligatures=TeX,Scale=MatchLowercase}
\fi
% use upquote if available, for straight quotes in verbatim environments
\IfFileExists{upquote.sty}{\usepackage{upquote}}{}
% use microtype if available
\IfFileExists{microtype.sty}{%
\usepackage{microtype}
\UseMicrotypeSet[protrusion]{basicmath} % disable protrusion for tt fonts
}{}
\usepackage[margin=1in]{geometry}
\usepackage{hyperref}
\hypersetup{unicode=true,
            pdftitle={Homework 8},
            pdfauthor={Kristiyan Dimitrov, Jieda Li, Parth Patel, Kristian Nikolov},
            pdfborder={0 0 0},
            breaklinks=true}
\urlstyle{same}  % don't use monospace font for urls
\usepackage{color}
\usepackage{fancyvrb}
\newcommand{\VerbBar}{|}
\newcommand{\VERB}{\Verb[commandchars=\\\{\}]}
\DefineVerbatimEnvironment{Highlighting}{Verbatim}{commandchars=\\\{\}}
% Add ',fontsize=\small' for more characters per line
\usepackage{framed}
\definecolor{shadecolor}{RGB}{248,248,248}
\newenvironment{Shaded}{\begin{snugshade}}{\end{snugshade}}
\newcommand{\AlertTok}[1]{\textcolor[rgb]{0.94,0.16,0.16}{#1}}
\newcommand{\AnnotationTok}[1]{\textcolor[rgb]{0.56,0.35,0.01}{\textbf{\textit{#1}}}}
\newcommand{\AttributeTok}[1]{\textcolor[rgb]{0.77,0.63,0.00}{#1}}
\newcommand{\BaseNTok}[1]{\textcolor[rgb]{0.00,0.00,0.81}{#1}}
\newcommand{\BuiltInTok}[1]{#1}
\newcommand{\CharTok}[1]{\textcolor[rgb]{0.31,0.60,0.02}{#1}}
\newcommand{\CommentTok}[1]{\textcolor[rgb]{0.56,0.35,0.01}{\textit{#1}}}
\newcommand{\CommentVarTok}[1]{\textcolor[rgb]{0.56,0.35,0.01}{\textbf{\textit{#1}}}}
\newcommand{\ConstantTok}[1]{\textcolor[rgb]{0.00,0.00,0.00}{#1}}
\newcommand{\ControlFlowTok}[1]{\textcolor[rgb]{0.13,0.29,0.53}{\textbf{#1}}}
\newcommand{\DataTypeTok}[1]{\textcolor[rgb]{0.13,0.29,0.53}{#1}}
\newcommand{\DecValTok}[1]{\textcolor[rgb]{0.00,0.00,0.81}{#1}}
\newcommand{\DocumentationTok}[1]{\textcolor[rgb]{0.56,0.35,0.01}{\textbf{\textit{#1}}}}
\newcommand{\ErrorTok}[1]{\textcolor[rgb]{0.64,0.00,0.00}{\textbf{#1}}}
\newcommand{\ExtensionTok}[1]{#1}
\newcommand{\FloatTok}[1]{\textcolor[rgb]{0.00,0.00,0.81}{#1}}
\newcommand{\FunctionTok}[1]{\textcolor[rgb]{0.00,0.00,0.00}{#1}}
\newcommand{\ImportTok}[1]{#1}
\newcommand{\InformationTok}[1]{\textcolor[rgb]{0.56,0.35,0.01}{\textbf{\textit{#1}}}}
\newcommand{\KeywordTok}[1]{\textcolor[rgb]{0.13,0.29,0.53}{\textbf{#1}}}
\newcommand{\NormalTok}[1]{#1}
\newcommand{\OperatorTok}[1]{\textcolor[rgb]{0.81,0.36,0.00}{\textbf{#1}}}
\newcommand{\OtherTok}[1]{\textcolor[rgb]{0.56,0.35,0.01}{#1}}
\newcommand{\PreprocessorTok}[1]{\textcolor[rgb]{0.56,0.35,0.01}{\textit{#1}}}
\newcommand{\RegionMarkerTok}[1]{#1}
\newcommand{\SpecialCharTok}[1]{\textcolor[rgb]{0.00,0.00,0.00}{#1}}
\newcommand{\SpecialStringTok}[1]{\textcolor[rgb]{0.31,0.60,0.02}{#1}}
\newcommand{\StringTok}[1]{\textcolor[rgb]{0.31,0.60,0.02}{#1}}
\newcommand{\VariableTok}[1]{\textcolor[rgb]{0.00,0.00,0.00}{#1}}
\newcommand{\VerbatimStringTok}[1]{\textcolor[rgb]{0.31,0.60,0.02}{#1}}
\newcommand{\WarningTok}[1]{\textcolor[rgb]{0.56,0.35,0.01}{\textbf{\textit{#1}}}}
\usepackage{graphicx,grffile}
\makeatletter
\def\maxwidth{\ifdim\Gin@nat@width>\linewidth\linewidth\else\Gin@nat@width\fi}
\def\maxheight{\ifdim\Gin@nat@height>\textheight\textheight\else\Gin@nat@height\fi}
\makeatother
% Scale images if necessary, so that they will not overflow the page
% margins by default, and it is still possible to overwrite the defaults
% using explicit options in \includegraphics[width, height, ...]{}
\setkeys{Gin}{width=\maxwidth,height=\maxheight,keepaspectratio}
\IfFileExists{parskip.sty}{%
\usepackage{parskip}
}{% else
\setlength{\parindent}{0pt}
\setlength{\parskip}{6pt plus 2pt minus 1pt}
}
\setlength{\emergencystretch}{3em}  % prevent overfull lines
\providecommand{\tightlist}{%
  \setlength{\itemsep}{0pt}\setlength{\parskip}{0pt}}
\setcounter{secnumdepth}{0}
% Redefines (sub)paragraphs to behave more like sections
\ifx\paragraph\undefined\else
\let\oldparagraph\paragraph
\renewcommand{\paragraph}[1]{\oldparagraph{#1}\mbox{}}
\fi
\ifx\subparagraph\undefined\else
\let\oldsubparagraph\subparagraph
\renewcommand{\subparagraph}[1]{\oldsubparagraph{#1}\mbox{}}
\fi

%%% Use protect on footnotes to avoid problems with footnotes in titles
\let\rmarkdownfootnote\footnote%
\def\footnote{\protect\rmarkdownfootnote}

%%% Change title format to be more compact
\usepackage{titling}

% Create subtitle command for use in maketitle
\providecommand{\subtitle}[1]{
  \posttitle{
    \begin{center}\large#1\end{center}
    }
}

\setlength{\droptitle}{-2em}

  \title{Homework 8}
    \pretitle{\vspace{\droptitle}\centering\huge}
  \posttitle{\par}
    \author{Kristiyan Dimitrov, Jieda Li, Parth Patel, Kristian Nikolov}
    \preauthor{\centering\large\emph}
  \postauthor{\par}
      \predate{\centering\large\emph}
  \postdate{\par}
    \date{11/29/2019}


\begin{document}
\maketitle

\hypertarget{exercise-8.5}{%
\subsection{Exercise 8.5}\label{exercise-8.5}}

\hypertarget{a}{%
\subsubsection{a)}\label{a}}

\begin{Shaded}
\begin{Highlighting}[]
\CommentTok{# Import Data}
\NormalTok{iris =}\StringTok{ }\KeywordTok{read.csv}\NormalTok{(}\StringTok{"/Users/kristiyan/Documents/MSiA 401 - Predictive 1/Datasets/Iris.csv"}\NormalTok{)}
\KeywordTok{str}\NormalTok{(iris)}
\end{Highlighting}
\end{Shaded}

\begin{verbatim}
## 'data.frame':    150 obs. of  6 variables:
##  $ Species_No  : int  1 1 1 1 1 1 1 1 1 1 ...
##  $ Petal_width : num  0.2 0.2 0.2 0.2 0.2 0.4 0.3 0.2 0.2 0.1 ...
##  $ Petal_length: num  1.4 1.4 1.3 1.5 1.4 1.7 1.4 1.5 1.4 1.5 ...
##  $ Sepal_width : num  3.5 3 3.2 3.1 3.6 3.9 3.4 3.4 2.9 3.1 ...
##  $ Sepal_length: num  5.1 4.9 4.7 4.6 5 5.4 4.6 5 4.4 4.9 ...
##  $ Species_name: Factor w/ 3 levels " Setosa"," Verginica",..: 1 1 1 1 1 1 1 1 1 1 ...
\end{verbatim}

\begin{Shaded}
\begin{Highlighting}[]
\KeywordTok{library}\NormalTok{(MASS)}
\NormalTok{irisLDA =}\StringTok{ }\KeywordTok{lda}\NormalTok{(Species_name }\OperatorTok{~}\StringTok{ }\NormalTok{Petal_width }\OperatorTok{+}\StringTok{ }\NormalTok{Petal_length }\OperatorTok{+}\StringTok{ }\NormalTok{Sepal_width }\OperatorTok{+}\StringTok{ }\NormalTok{Sepal_length, }\DataTypeTok{data =}\NormalTok{ iris, }\DataTypeTok{prior=}\KeywordTok{c}\NormalTok{(}\DecValTok{1}\NormalTok{,}\DecValTok{1}\NormalTok{,}\DecValTok{1}\NormalTok{)}\OperatorTok{/}\DecValTok{3}\NormalTok{) }\CommentTok{# Calculate Fisher's Linear Discriminant Functions}
\NormalTok{irisLDA }\CommentTok{# At the very end of the output we can see that most of the separability is captured by the first Linear Discriminant Function (LD1 - 99.12%)}
\end{Highlighting}
\end{Shaded}

\begin{verbatim}
## Call:
## lda(Species_name ~ Petal_width + Petal_length + Sepal_width + 
##     Sepal_length, data = iris, prior = c(1, 1, 1)/3)
## 
## Prior probabilities of groups:
##      Setosa   Verginica  Versicolor 
##   0.3333333   0.3333333   0.3333333 
## 
## Group means:
##             Petal_width Petal_length Sepal_width Sepal_length
##  Setosa           0.246        1.462       3.428        5.006
##  Verginica        2.026        5.552       2.974        6.588
##  Versicolor       1.326        4.260       2.770        5.936
## 
## Coefficients of linear discriminants:
##                     LD1         LD2
## Petal_width  -2.8104603 -2.83918785
## Petal_length -2.2012117  0.93192121
## Sepal_width   1.5344731 -2.16452123
## Sepal_length  0.8293776 -0.02410215
## 
## Proportion of trace:
##    LD1    LD2 
## 0.9912 0.0088
\end{verbatim}

\begin{Shaded}
\begin{Highlighting}[]
\NormalTok{irisLDA[}\DecValTok{4}\NormalTok{] }\CommentTok{# These are the coefficients of the Fisher Discriminant functions}
\end{Highlighting}
\end{Shaded}

\begin{verbatim}
## $scaling
##                     LD1         LD2
## Petal_width  -2.8104603 -2.83918785
## Petal_length -2.2012117  0.93192121
## Sepal_width   1.5344731 -2.16452123
## Sepal_length  0.8293776 -0.02410215
\end{verbatim}

\hypertarget{b}{%
\subsubsection{b)}\label{b}}

\begin{Shaded}
\begin{Highlighting}[]
\KeywordTok{predict}\NormalTok{( irisLDA, }\DataTypeTok{newdata =} \KeywordTok{data.frame}\NormalTok{(}\DataTypeTok{Sepal_width =} \DecValTok{3}\NormalTok{, }\DataTypeTok{Sepal_length =} \FloatTok{5.5}\NormalTok{, }\DataTypeTok{Petal_width =} \FloatTok{1.5}\NormalTok{, }\DataTypeTok{Petal_length =} \DecValTok{4}\NormalTok{))}
\end{Highlighting}
\end{Shaded}

\begin{verbatim}
## $class
## [1]  Versicolor
## Levels:  Setosa  Verginica  Versicolor
## 
## $posterior
##         Setosa    Verginica  Versicolor
## 1 1.962381e-19 0.0006246577   0.9993753
## 
## $x
##         LD1        LD2
## 1 -1.750434 -0.4957499
\end{verbatim}

We see that Versicolor has the highest probability therefore we would
classify this observation as Versicolor

\hypertarget{exercise-9.3}{%
\subsection{Exercise 9.3}\label{exercise-9.3}}

\begin{Shaded}
\begin{Highlighting}[]
\CommentTok{# Import Data}
\NormalTok{injury =}\StringTok{ }\KeywordTok{read.csv}\NormalTok{(}\StringTok{"/Users/kristiyan/Documents/MSiA 401 - Predictive 1/Datasets/Airline-Injury.csv"}\NormalTok{)}
\KeywordTok{str}\NormalTok{(injury)}
\end{Highlighting}
\end{Shaded}

\begin{verbatim}
## 'data.frame':    9 obs. of  2 variables:
##  $ x: num  0.095 0.192 0.075 0.208 0.138 ...
##  $ y: int  11 7 7 19 9 4 3 1 3
\end{verbatim}

\begin{Shaded}
\begin{Highlighting}[]
\KeywordTok{plot}\NormalTok{(injury)}
\end{Highlighting}
\end{Shaded}

\includegraphics{Homework-8_files/figure-latex/unnamed-chunk-6-1.pdf}

\hypertarget{part-1---simple-linear-regression-with-no-transformation-on-y}{%
\subsubsection{Part 1 - Simple linear regression with no transformation
on
y}\label{part-1---simple-linear-regression-with-no-transformation-on-y}}

\begin{Shaded}
\begin{Highlighting}[]
\CommentTok{# Using lm() function}
\NormalTok{lmfit =}\StringTok{ }\KeywordTok{lm}\NormalTok{(y }\OperatorTok{~}\StringTok{ }\NormalTok{x, }\DataTypeTok{data =}\NormalTok{ injury)}
\KeywordTok{summary}\NormalTok{(lmfit)}
\end{Highlighting}
\end{Shaded}

\begin{verbatim}
## 
## Call:
## lm(formula = y ~ x, data = injury)
## 
## Residuals:
##     Min      1Q  Median      3Q     Max 
## -5.3351 -2.1281  0.1605  2.2670  5.6382 
## 
## Coefficients:
##             Estimate Std. Error t value Pr(>|t|)  
## (Intercept)  -0.1402     3.1412  -0.045   0.9657  
## x            64.9755    25.1959   2.579   0.0365 *
## ---
## Signif. codes:  0 '***' 0.001 '**' 0.01 '*' 0.05 '.' 0.1 ' ' 1
## 
## Residual standard error: 4.201 on 7 degrees of freedom
## Multiple R-squared:  0.4872, Adjusted R-squared:  0.4139 
## F-statistic:  6.65 on 1 and 7 DF,  p-value: 0.03654
\end{verbatim}

\begin{Shaded}
\begin{Highlighting}[]
\CommentTok{# We check the SSE for the lm model}
\KeywordTok{anova}\NormalTok{(lmfit)[}\DecValTok{2}\NormalTok{,}\DecValTok{2}\NormalTok{]}
\end{Highlighting}
\end{Shaded}

\begin{verbatim}
## [1] 123.5302
\end{verbatim}

\begin{Shaded}
\begin{Highlighting}[]
\CommentTok{# Verifying SSE by direct calculation}
\KeywordTok{sum}\NormalTok{(lmfit}\OperatorTok{$}\NormalTok{residuals}\OperatorTok{^}\DecValTok{2}\NormalTok{)}
\end{Highlighting}
\end{Shaded}

\begin{verbatim}
## [1] 123.5302
\end{verbatim}

\begin{Shaded}
\begin{Highlighting}[]
\CommentTok{# Using glm() function}
\NormalTok{linearRegression =}\StringTok{ }\KeywordTok{glm}\NormalTok{(y }\OperatorTok{~}\StringTok{ }\NormalTok{x, }\DataTypeTok{data =}\NormalTok{ injury, }\DataTypeTok{family =}\NormalTok{ gaussian)}
\KeywordTok{summary}\NormalTok{(linearRegression)}
\end{Highlighting}
\end{Shaded}

\begin{verbatim}
## 
## Call:
## glm(formula = y ~ x, family = gaussian, data = injury)
## 
## Deviance Residuals: 
##     Min       1Q   Median       3Q      Max  
## -5.3351  -2.1281   0.1605   2.2670   5.6382  
## 
## Coefficients:
##             Estimate Std. Error t value Pr(>|t|)  
## (Intercept)  -0.1402     3.1412  -0.045   0.9657  
## x            64.9755    25.1959   2.579   0.0365 *
## ---
## Signif. codes:  0 '***' 0.001 '**' 0.01 '*' 0.05 '.' 0.1 ' ' 1
## 
## (Dispersion parameter for gaussian family taken to be 17.64717)
## 
##     Null deviance: 240.89  on 8  degrees of freedom
## Residual deviance: 123.53  on 7  degrees of freedom
## AIC: 55.114
## 
## Number of Fisher Scoring iterations: 2
\end{verbatim}

\begin{Shaded}
\begin{Highlighting}[]
\CommentTok{# the Deviance for the linear model is actually SSE. We see the value is once again confirmed}
\NormalTok{linearRegression}\OperatorTok{$}\NormalTok{deviance}
\end{Highlighting}
\end{Shaded}

\begin{verbatim}
## [1] 123.5302
\end{verbatim}

\hypertarget{part-2---linear-regression-of-sqrt-transform-of-y}{%
\section{Part 2 - Linear Regression of sqrt() transform of
y}\label{part-2---linear-regression-of-sqrt-transform-of-y}}

\begin{Shaded}
\begin{Highlighting}[]
\NormalTok{lmfitSqrt =}\StringTok{ }\KeywordTok{lm}\NormalTok{(}\KeywordTok{sqrt}\NormalTok{(y) }\OperatorTok{~}\StringTok{ }\NormalTok{x, }\DataTypeTok{data =}\NormalTok{ injury)}
\KeywordTok{summary}\NormalTok{(lmfitSqrt)}
\end{Highlighting}
\end{Shaded}

\begin{verbatim}
## 
## Call:
## lm(formula = sqrt(y) ~ x, data = injury)
## 
## Residuals:
##     Min      1Q  Median      3Q     Max 
## -0.9690 -0.7655  0.1906  0.5874  1.0211 
## 
## Coefficients:
##             Estimate Std. Error t value Pr(>|t|)  
## (Intercept)   1.1692     0.5783   2.022   0.0829 .
## x            11.8564     4.6382   2.556   0.0378 *
## ---
## Signif. codes:  0 '***' 0.001 '**' 0.01 '*' 0.05 '.' 0.1 ' ' 1
## 
## Residual standard error: 0.7733 on 7 degrees of freedom
## Multiple R-squared:  0.4828, Adjusted R-squared:  0.4089 
## F-statistic: 6.535 on 1 and 7 DF,  p-value: 0.03776
\end{verbatim}

\begin{Shaded}
\begin{Highlighting}[]
\NormalTok{linearRegressionSqrtSSE =}\StringTok{ }\KeywordTok{sum}\NormalTok{((injury}\OperatorTok{$}\NormalTok{y }\OperatorTok{-}\StringTok{ }\NormalTok{lmfitSqrt}\OperatorTok{$}\NormalTok{fitted.values}\OperatorTok{^}\DecValTok{2}\NormalTok{)}\OperatorTok{^}\DecValTok{2}\NormalTok{)}
\NormalTok{linearRegressionSqrtSSE}
\end{Highlighting}
\end{Shaded}

\begin{verbatim}
## [1] 123.0247
\end{verbatim}

We see that the SSE for the sqrt() transformed y appears to be only
slightly lower than the regular lm model. Furthermore, the variables are
still reasonably significant. The plot below clearly shows that a linear
fit is not the best idea. This is also supported by the low R\^{}2 we
get for both models.

\begin{Shaded}
\begin{Highlighting}[]
\KeywordTok{plot}\NormalTok{(injury}\OperatorTok{$}\NormalTok{x, }\KeywordTok{sqrt}\NormalTok{(injury}\OperatorTok{$}\NormalTok{y) )}
\end{Highlighting}
\end{Shaded}

\includegraphics{Homework-8_files/figure-latex/unnamed-chunk-14-1.pdf}

\hypertarget{part-3---poisson-regression}{%
\subsubsection{Part 3 - Poisson
Regression}\label{part-3---poisson-regression}}

\begin{Shaded}
\begin{Highlighting}[]
\NormalTok{poissonRegression =}\StringTok{ }\KeywordTok{glm}\NormalTok{(y }\OperatorTok{~}\StringTok{ }\NormalTok{x  , }\DataTypeTok{data =}\NormalTok{ injury, }\DataTypeTok{family =}\NormalTok{ poisson)}
\KeywordTok{summary}\NormalTok{(poissonRegression)}
\end{Highlighting}
\end{Shaded}

\begin{verbatim}
## 
## Call:
## glm(formula = y ~ x, family = poisson, data = injury)
## 
## Deviance Residuals: 
##      Min        1Q    Median        3Q       Max  
## -1.81894  -1.69082   0.06495   1.02407   2.06811  
## 
## Coefficients:
##             Estimate Std. Error z value Pr(>|z|)    
## (Intercept)   0.8945     0.3265   2.739  0.00615 ** 
## x             8.5018     2.1575   3.941 8.13e-05 ***
## ---
## Signif. codes:  0 '***' 0.001 '**' 0.01 '*' 0.05 '.' 0.1 ' ' 1
## 
## (Dispersion parameter for poisson family taken to be 1)
## 
##     Null deviance: 31.859  on 8  degrees of freedom
## Residual deviance: 16.291  on 7  degrees of freedom
## AIC: 52.251
## 
## Number of Fisher Scoring iterations: 5
\end{verbatim}

\begin{Shaded}
\begin{Highlighting}[]
\NormalTok{poissonRegressionSSE =}\StringTok{ }\KeywordTok{sum}\NormalTok{((poissonRegression}\OperatorTok{$}\NormalTok{fitted.values }\OperatorTok{-}\StringTok{ }\NormalTok{injury}\OperatorTok{$}\NormalTok{y)}\OperatorTok{^}\DecValTok{2}\NormalTok{)}
\NormalTok{poissonRegressionSSE}
\end{Highlighting}
\end{Shaded}

\begin{verbatim}
## [1] 117.3472
\end{verbatim}

This is the lowest SSE model and also the model we'd prefer. - The
variables are much more significant - The overall model deviance passes
the chi-squared significance test

\hypertarget{exercise-9.4}{%
\subsection{Exercise 9.4}\label{exercise-9.4}}

\begin{Shaded}
\begin{Highlighting}[]
\CommentTok{# Import Data}
\NormalTok{crash =}\StringTok{ }\KeywordTok{read.csv}\NormalTok{(}\StringTok{"/Users/kristiyan/Documents/MSiA 401 - Predictive 1/Datasets/crashdata2014-summary.csv"}\NormalTok{)}
\NormalTok{crash <-}\StringTok{ }\NormalTok{crash[}\OperatorTok{-}\DecValTok{1}\NormalTok{] }\CommentTok{# Dropping the index}
\KeywordTok{str}\NormalTok{(crash)}
\end{Highlighting}
\end{Shaded}

\begin{verbatim}
## 'data.frame':    1152 obs. of  8 variables:
##  $ Day           : Factor w/ 2 levels "Weekday","Weekend": 1 2 1 2 1 2 1 2 1 2 ...
##  $ Time          : Factor w/ 4 levels "Evening","Midday",..: 4 4 3 3 2 2 1 1 4 4 ...
##  $ TrafficControl: Factor w/ 3 levels "Control","No Control",..: 2 2 2 2 2 2 2 2 1 1 ...
##  $ Road          : Factor w/ 3 levels "Dry","Other",..: 1 1 1 1 1 1 1 1 1 1 ...
##  $ Light         : Factor w/ 4 levels "Dark","Dawn/Dusk",..: 3 3 3 3 3 3 3 3 3 3 ...
##  $ Weather       : Factor w/ 4 levels "Clear","Other",..: 1 1 1 1 1 1 1 1 1 1 ...
##  $ Weight        : int  5 2 5 2 5 2 5 2 5 2 ...
##  $ Count         : int  444 246 4147 951 5693 2193 3981 1210 239 121 ...
\end{verbatim}

\hypertarget{part-a---unweighted-poisson}{%
\subsubsection{Part a) - Unweighted
Poisson}\label{part-a---unweighted-poisson}}

\begin{Shaded}
\begin{Highlighting}[]
\NormalTok{unweightedPoisson =}\StringTok{ }\KeywordTok{glm}\NormalTok{(Count }\OperatorTok{~}\StringTok{ }\NormalTok{Day }\OperatorTok{+}\StringTok{ }\NormalTok{Time }\OperatorTok{+}\StringTok{ }\NormalTok{TrafficControl }\OperatorTok{+}\StringTok{ }\NormalTok{Road }\OperatorTok{+}\StringTok{ }\NormalTok{Light }\OperatorTok{+}\StringTok{ }\NormalTok{Weather, }\DataTypeTok{data =}\NormalTok{ crash, }\DataTypeTok{family =}\NormalTok{ poisson)}
\KeywordTok{summary}\NormalTok{(unweightedPoisson)}
\end{Highlighting}
\end{Shaded}

\begin{verbatim}
## 
## Call:
## glm(formula = Count ~ Day + Time + TrafficControl + Road + Light + 
##     Weather, family = poisson, data = crash)
## 
## Deviance Residuals: 
##     Min       1Q   Median       3Q      Max  
## -48.765   -4.197   -1.071    1.111   65.305  
## 
## Coefficients:
##                           Estimate Std. Error  z value Pr(>|z|)    
## (Intercept)               7.103091   0.010011  709.518   <2e-16 ***
## DayWeekend               -0.966953   0.007781 -124.266   <2e-16 ***
## TimeMidday               -0.008045   0.009227   -0.872    0.383    
## TimeMorning              -0.222000   0.009763  -22.738   <2e-16 ***
## TimeNight                -0.335498   0.010084  -33.270   <2e-16 ***
## TrafficControlNo Control  0.089846   0.007090   12.671   <2e-16 ***
## TrafficControlUnknown    -2.533891   0.018901 -134.063   <2e-16 ***
## RoadOther                -2.273723   0.013582 -167.410   <2e-16 ***
## RoadWet                  -1.134737   0.008412 -134.898   <2e-16 ***
## LightDawn/Dusk           -2.034986   0.019172 -106.142   <2e-16 ***
## LightDaylight             0.796045   0.007852  101.385   <2e-16 ***
## LightUnknown             -1.784394   0.017190 -103.806   <2e-16 ***
## WeatherOther             -2.587845   0.014941 -173.200   <2e-16 ***
## WeatherPoor Visibilty    -3.598460   0.024209 -148.644   <2e-16 ***
## WeatherRain/Snow         -1.665179   0.009906 -168.093   <2e-16 ***
## ---
## Signif. codes:  0 '***' 0.001 '**' 0.01 '*' 0.05 '.' 0.1 ' ' 1
## 
## (Dispersion parameter for poisson family taken to be 1)
## 
##     Null deviance: 428848  on 1151  degrees of freedom
## Residual deviance: 123720  on 1137  degrees of freedom
## AIC: 127219
## 
## Number of Fisher Scoring iterations: 7
\end{verbatim}

Comments on coefficients: - DayWeekend has negative coefficient, which
makes sense - fewer people drive on weekend --\textgreater{} fewer
crashes - TimeMidday is not signicicant --\textgreater{} no real
difference b/w Midday \& Evening, which makes sense , because the counts
are almost the same (23,399 \& 23,588) - Time Morning \& Time Night -
negative coefficients make sense, because there are fewer people driving
in morning \& night than in evening (which is the reference category) -
NoTrafficControl - positive coefficient makes sense - peple are more
likely to speed and therefore crash with no traffic control - RoadWet -
negative coefficient doesn't make much sense - would expect poor road
conditions to increase crashes - LightDawn/Dusk - negative coefficient
compared to reference category Dark doesn't make much sense. This
variable is very correlated with Time, which probably explains the
coefficients i.e.~signs are the same as Time Morning/Night -
LightDaylight - same as above - WeatherPoor Visibility \& Rain Snow -
correlated with RoadWet and the Light variables.

\hypertarget{part-b---weighted-poisson}{%
\subsubsection{Part b) - Weighted
Poisson}\label{part-b---weighted-poisson}}

\begin{Shaded}
\begin{Highlighting}[]
\NormalTok{weightedPoisson =}\StringTok{ }\KeywordTok{glm}\NormalTok{(Count }\OperatorTok{~}\StringTok{ }\NormalTok{Day }\OperatorTok{+}\StringTok{ }\NormalTok{Time }\OperatorTok{+}\StringTok{ }\NormalTok{TrafficControl }\OperatorTok{+}\StringTok{ }\NormalTok{Road }\OperatorTok{+}\StringTok{ }\NormalTok{Light }\OperatorTok{+}\StringTok{ }\NormalTok{Weather , }\DataTypeTok{data =}\NormalTok{ crash, }\DataTypeTok{weights =}\NormalTok{ Weight, }\DataTypeTok{family =}\NormalTok{ poisson)}
\KeywordTok{summary}\NormalTok{(weightedPoisson)}
\end{Highlighting}
\end{Shaded}

\begin{verbatim}
## 
## Call:
## glm(formula = Count ~ Day + Time + TrafficControl + Road + Light + 
##     Weather, family = poisson, data = crash, weights = Weight)
## 
## Deviance Residuals: 
##      Min        1Q    Median        3Q       Max  
## -103.052    -6.938    -1.792     2.190   133.894  
## 
## Coefficients:
##                           Estimate Std. Error  z value Pr(>|z|)    
## (Intercept)               7.061573   0.004882 1446.573  < 2e-16 ***
## DayWeekend               -0.966953   0.005027 -192.353  < 2e-16 ***
## TimeMidday               -0.030746   0.004488   -6.851 7.31e-12 ***
## TimeMorning              -0.179012   0.004666  -38.365  < 2e-16 ***
## TimeNight                -0.481346   0.005095  -94.477  < 2e-16 ***
## TrafficControlNo Control  0.067936   0.003468   19.590  < 2e-16 ***
## TrafficControlUnknown    -2.567926   0.009342 -274.886  < 2e-16 ***
## RoadOther                -2.281779   0.006684 -341.363  < 2e-16 ***
## RoadWet                  -1.116644   0.004097 -272.569  < 2e-16 ***
## LightDawn/Dusk           -1.961127   0.009426 -208.060  < 2e-16 ***
## LightDaylight             0.904702   0.003923  230.590  < 2e-16 ***
## LightUnknown             -1.757028   0.008629 -203.623  < 2e-16 ***
## WeatherOther             -2.610596   0.007394 -353.064  < 2e-16 ***
## WeatherPoor Visibilty    -3.568698   0.011684 -305.443  < 2e-16 ***
## WeatherRain/Snow         -1.655916   0.004832 -342.703  < 2e-16 ***
## ---
## Signif. codes:  0 '***' 0.001 '**' 0.01 '*' 0.05 '.' 0.1 ' ' 1
## 
## (Dispersion parameter for poisson family taken to be 1)
## 
##     Null deviance: 1764135  on 1151  degrees of freedom
## Residual deviance:  496975  on 1137  degrees of freedom
## AIC: 509909
## 
## Number of Fisher Scoring iterations: 7
\end{verbatim}

We note the following: - The weightedPoisson has onemore significant
variable (TimeMidday) - Has higher AIC - Has higher Residual Deviance

Overall, the weightedPoisson doesn't change the size of the coefficients
much, nore does it change their significance

\hypertarget{appendix}{%
\subsection{Appendix}\label{appendix}}

\begin{Shaded}
\begin{Highlighting}[]
\NormalTok{unweightedSSE =}\StringTok{ }\KeywordTok{sum}\NormalTok{((crash}\OperatorTok{$}\NormalTok{Count }\OperatorTok{-}\StringTok{ }\NormalTok{unweightedPoisson}\OperatorTok{$}\NormalTok{fitted.values)}\OperatorTok{^}\DecValTok{2}\NormalTok{)}
\NormalTok{unweightedSSE}
\end{Highlighting}
\end{Shaded}

\begin{verbatim}
## [1] 65068375
\end{verbatim}

\begin{Shaded}
\begin{Highlighting}[]
\NormalTok{weightedSSE =}\StringTok{ }\KeywordTok{sum}\NormalTok{((crash}\OperatorTok{$}\NormalTok{Count }\OperatorTok{-}\StringTok{ }\NormalTok{(weightedPoisson}\OperatorTok{$}\NormalTok{fitted.values)}\OperatorTok{*}\NormalTok{(crash}\OperatorTok{$}\NormalTok{Weight))}\OperatorTok{^}\DecValTok{2}\NormalTok{)}
\NormalTok{weightedSSE}
\end{Highlighting}
\end{Shaded}

\begin{verbatim}
## [1] 1096273154
\end{verbatim}


\end{document}
